\documentclass[modern]{aastex631}
\usepackage{showyourwork}

\newcommand{\obs}{\mathrm{obs}}

\newcommand{\cca}{Center for Computational Astrophysics, Flatiron Institute, New York NY 10010, USA}
\newcommand{\sbu}{Department of Physics and Astronomy, Stony Brook University, Stony Brook NY 11794, USA}

% Begin!
\begin{document}

% Title
\title{Eliminating Systematic Bias in Bright-Siren Multimessenger Observations of Compact Object Mergers from Inclination-Dependent Emission}

% Author list
\author{Will M. Farr}
\email{will.farr@stonybrook.edu}
\affiliation{\sbu}
\affiliation{\cca}


% Abstract with filler text
\begin{abstract}
    This article and associated code explores systematics associated with
inclination dependence in detection of bright counterparts to GW mergers.  The
TL;DR is that, by fitting out the inclination dependence of EM detections with
fairly simple models, we can remove what otherwise would be a bias that could
prevent percent-level expansion history measurements with bright sirens.
\end{abstract}

% Main body with filler text
\section{Introduction}
\label{sec:intro}

\section{Observational Model}

We use a simplified observational model that reproduces the essential aspects of
gravitational wave distance measurements without requiring generating and
inferring full gravitational waveforms.  We imagine that each gravitational wave
event has a (cosine) inclination $x = \cos \iota$ and a distance, and these
combine to give an amplitude in the left- and right-handed polarization
components of 
\begin{equation}
    A_{R/L} = \frac{\left( 1 \pm x \right)^2}{d}
\end{equation}
in some units.  We assume that a detector network measures a linear projection
of these these polarization amplitudes with unit-amplitude Gaussian noise:
\begin{equation}
    A_{R/L,\obs} = f_{R/L} A_{R/L} + N(0,1).
\end{equation}
For a detector network with two or more arbitrarily oriented, nearly
equally-sensitive detectors, we can set $f_R = f_L = 1$; for two nearly-aligned
equally-sensitive detectors (e.g.\ the two LIGO detectors in Washington and
Louisiana), we can set $f_R \ll f_L = 1$ to reflect that one polarization
component is measured much more precisely than the other.  From here onward we assume events used for bright sirens have small localization volumes and are therefore observed in three or more detectors and we set $f_L = f_R = 1$.

The S/N of an observation in our model is given by 
\begin{equation}
    \rho^2 = \frac{A_{R,\obs}^2 + A_{L,\obs}^2}{2}.
\end{equation}
We impose a Euclidean prior on $d$, $p(d) \propto d^2$, appropriate for the
nearby universe, and a flat prior on $x$, $p(x) = 1/2$, appropriate for an
intrinsically isotropic population of GW mergers.  Inferring $d$ and $x$ from
$A_{R,\obs}$ and $A_{L,\obs}$ in our model results in posterior densities that
reproduce the distance-inclination correlation observed in analysis of actual GW
events, as well as the (fractional) distance uncertainty and inclination
uncertainty observed in actual GW events at comparable S/N.  Figure
\ref{fig:distance-iota} shows the inferred distance and inclination from our
model for a GW170817-like event ($\rho \simeq 30$, $x \simeq -1$); our model's
inferences compare directly to those for GW170817 itself \citep{Abbott2017}.

\begin{figure}
    \script{distance-iota.py}
    \includegraphics[width=\columnwidth]{figures/distance-iota.pdf}
    \caption{\label{fig:distance-iota} Posterior on distance $d$ and $x = \cos
    \iota$ for a GW170817-like event ($\rho \simeq 30$) from our observational
    model.  Contour levels correspond to succesive 10\% credible intervals; true
    parameters are indicated by the black lines.  Compare to Figure 2 of
    \citet{Abbott2017}.  The fractional uncertainty on $d$ after marginalizing
    over $x$ with an isotropic (flat) prior is $\sigma_d / \left\langle d
    \right\rangle \simeq 0.14$; for GW170817 the same quantity is also $\sigma_d
    / \left\langle d \right\rangle \simeq \sigma_{H_0} / \left\langle H_0
    \right\rangle \simeq 0.14$ \citep{Abbott2017}.  Similarly, our model has
    $x_{84\%} \simeq -0.82$, while for GW170817 the equivalent 1-$\sigma$ upper
    limit on $x \simeq -0.81$ \citep{Abbott2017}.  Our model thus does a
    reasonable job reproducing the correlation between distance and inclination
    and the relevant uncertainties at the S/N of GW170817; in both our model and
    a full analysis of a GW event uncertainties will scale inversely with S/N.}
\end{figure}

\subsection{Effect of EM Selection}

\begin{acknowledgments}
    We thank [people] for conversations.
\end{acknowledgments}

\software{\texttt{arviz} \citep{arviz_2019}, \texttt{matplotlib} \citep{Hunter:2007}, \texttt{pymc} \citep{salvatier2016probabilistic}, \texttt{seaborn} \citep{Waskom2021}, \texttt{showyourwork} \citep{Luger2021}}

\bibliography{bib}

\end{document}
