\documentclass[modern]{aastex631}
\usepackage{showyourwork}

\newcommand{\todo}[1]{\textcolor{blue}{TODO: #1}}

\newcommand{\dd}{\mathrm{d}}

\newcommand{\detected}{\mathrm{det}}
\newcommand{\EM}{\mathrm{EM}}
\newcommand{\GW}{\mathrm{GW}}
\newcommand{\obs}{\mathrm{obs}}

\newcommand{\cca}{Center for Computational Astrophysics, Flatiron Institute, New York NY 10010, USA}
\newcommand{\sbu}{Department of Physics and Astronomy, Stony Brook University, Stony Brook NY 11794, USA}

% Begin!
\begin{document}

% Title
\title{Eliminating Systematic Bias in Bright-Siren Multimessenger Observations of Compact Object Mergers from Inclination-Dependent Emission}

% Author list
\author{Will M. Farr}
\email{will.farr@stonybrook.edu}
\affiliation{\sbu}
\affiliation{\cca}

\author{Friends}


% Abstract with filler text
\begin{abstract}
    This article and associated code explores systematics associated with
inclination dependence in detection of bright counterparts to GW mergers.  The
TL;DR is that, by fitting out the inclination dependence of EM detections with
fairly simple models, we can remove what otherwise would be a bias that could
prevent percent-level expansion history measurements with bright sirens.
\end{abstract}

% Main body with filler text
\section{Introduction}
\label{sec:intro}

\section{Observational Model}

We use a simplified observational model that reproduces the essential aspects of
gravitational wave distance measurements without requiring generating and
inferring full gravitational waveforms.  We imagine that each gravitational wave
event has a (cosine) inclination $x = \cos \iota$ and a distance, and these
combine to give an amplitude in the left- and right-handed polarization
components of 
\begin{equation}
    A_{R/L} = \frac{\left( 1 \pm x \right)^2}{d}
\end{equation}
in some units.  We assume that a detector network measures a linear projection
of these these polarization amplitudes with unit-amplitude Gaussian noise:
\begin{equation}
    A_{R/L,\obs} = f_{R/L} A_{R/L} + N(0,1).
\end{equation}
For a detector network with two or more arbitrarily oriented, nearly
equally-sensitive detectors, we can set $f_R = f_L = 1$; for two nearly-aligned
equally-sensitive detectors (e.g.\ the two LIGO detectors in Washington and
Louisiana), we can set $f_R \ll f_L = 1$ to reflect that one polarization
component is measured much more precisely than the other.  From here onward we assume events used for bright sirens have small localization volumes and are therefore observed in three or more detectors and we set $f_L = f_R = 1$.

The S/N of an observation in our model is given by 
\begin{equation}
    \rho^2 = \frac{A_{R,\obs}^2 + A_{L,\obs}^2}{2}.
\end{equation}
We impose a Euclidean prior on $d$, $p(d) \propto d^2$, appropriate for the
nearby universe, and a flat prior on $x$, $p(x) = 1/2$, appropriate for an
intrinsically isotropic population of GW mergers.  Inferring $d$ and $x$ from
$A_{R,\obs}$ and $A_{L,\obs}$ in our model results in posterior densities that
reproduce the distance-inclination correlation observed in analysis of actual GW
events, as well as the (fractional) distance uncertainty and inclination
uncertainty observed in actual GW events at comparable S/N.  Figure
\ref{fig:distance-iota} shows the inferred distance and inclination from our
model for a GW170817-like event ($\rho \simeq 30$, $x \simeq -1$); our model's
inferences compare directly to those for GW170817 itself \citep{Abbott2017}.

\begin{figure}
    \script{distance-iota.py}
    \includegraphics[width=\columnwidth]{figures/distance-iota.pdf}
    \caption{Posterior on distance $d$ and $x = \cos \iota$ for a GW170817-like
    event ($\rho \simeq 30$) from our observational model.  Contour levels
    correspond to succesive 10\% credible intervals; true parameters are
    indicated by the black lines.  Compare to Figure 2 of \citet{Abbott2017}.
    The fractional uncertainty on $d$ after marginalizing over $x$ with an
    isotropic (flat) prior is $\sigma_d / \left\langle d \right\rangle \simeq
    0.14$; for GW170817 the same quantity is also $\sigma_d / \left\langle d
    \right\rangle \simeq \sigma_{H_0} / \left\langle H_0 \right\rangle \simeq
    0.14$ \citep{Abbott2017}.  Similarly, our model has $x_{84\%} \simeq -0.82$,
    while for GW170817 the equivalent 1-$\sigma$ upper limit on $x \simeq -0.81$
    \citep{Abbott2017}.  Our model thus does a reasonable job reproducing the
    correlation between distance and inclination and the relevant uncertainties
    at the S/N of GW170817; in both our model and a full analysis of a GW event
    uncertainties will scale inversely with S/N.}
    \label{fig:distance-iota}
\end{figure}

\subsection{Effect of EM Selection}

We assume that each of our observations also has a detection of an
electromagnetic counterpart that gives a perfect measurement of redshift, thus
constraining $H_0$.  The full likelihood for both GW and EM data for an observation factorizes because the measurement noise for the observations are independent:
\begin{equation}
    p\left( d_\GW, d_\EM \mid x, d, H_0 \right) = p\left( d_\GW \mid x, d \right) p\left( d_\EM \mid x, d, H_0 \right).
\end{equation}
The first term is the likelhood representing the measurement model described
above; the second term relates the EM measurements to the inclination, distance,
and (through the distance-redshift relation) $H_0$.  If we were willing to model
the dependence of the EM data on $x$, we could use it to further constrain the
inclination measurement and thereby improve the $H_0$ measurement.  But here we
suppose that this is not possible---either we feel that the inclination modeling
for the EM data is systematics-dominated, or we are unable to even produce a
model of the inclination-dependent emission being measured.  In this case we can
marginalize the EM data out of the likelihood, retaining only (1) the redshift
measurement and (2) the knowledge that the EM emission was detectable.  Let $\Omega$ be the set of EM data consistent with these criteria.  Then
\begin{equation}
    \int_{d_\EM \in \Omega} \dd d_\EM \, p\left( d_\EM \mid x, d, H_0 \right) \propto \delta\left( z - H_0 d \right) P_\mathrm{det,\EM}\left( x, d, H_0 \right),
\end{equation}
where $P_{\detected,\EM}\left(x, d, H_0\right)$ is the fraction of mergers with
inclination $x$, at distance $d$, and with Hubble constant $H_0$ that produce
detectable EM emission \citep{Mandel2019}.

If we had a good model for the EM emission, we could \emph{calculate}
$P_{\detected,\EM}$; but if we do not, we can treat it as a parameterized
function, and learn its shape from the GW data.  (Intuitively: the GW data carry
some information about the inclination of each event; considering the set of
events with coincident EM detections lets us learn the population of
inclinations for \emph{just these} events, which is proportional to
$P_{\detected,\EM}$.)  The term $P_{\detected,\EM}$ acts like a modification of
the prior applied to $x$, $d$, and $H_0$ in the overall analysis.  After
integrating out the EM data in $\Omega$, we obtain a modified GW likelihood:
\begin{multline}
    p\left( d_\GW, z \mid x, d, H_0 \right) = \int_{d_\EM \in \Omega} \dd d_\EM p\left( d_\GW \mid x, d \right) p\left( d_\EM \mid x, d, H_0 \right) \\ \propto \delta\left( z - H_0 d \right) p\left( d_\GW \mid x, d \right) P_{\detected,\EM}\left( x, d, H_0 \right).
\end{multline}

Applying a $d^2$ prior and integrating across the $\delta$ function, we obtain 
\begin{multline}
    p\left( d_\GW, z \mid x, H_0 \right) \propto \int \dd d \, p\left( d_\GW, z \mid x, d, H_0 \right) d^2 \\ = \frac{z^2}{H_0^3} P_{\detected,\EM}\left( x, d = z/H_0, H_0 \right) p\left( d_\GW \mid x, d=z/H_0 \right).
\end{multline}
Here we see that $P_{\detected,\EM}$ acts as an additional multiplicative
factor, appearing as a modification to the isotropic prior that we would
naturally apply for the inclination of our gravitational wave population.

\todo{Flesh this out, and discuss GW selection (mechanical, and our model, with
$\rho > 10$); and then get specific about our model for $P_{\detected,\EM}$.
Also discuss ``joint'' catalogs, like combining GRB-selected and
kilonova-selected data sets.}

\subsection{Functional Form of $P_{\detected,\EM}$}

Here for simplicity we assume 
\begin{itemize}
    \item $P_{\detected,\EM}$ does not depend on $d$ wherever GW events are
    detectable.  That is: the distance-dependence of selection effects is
    dominated by the GW selection-function.
    \item $P_{\detected,\EM}$ is independent of $H_0$.  This will be the case
    whenever the choice of EM survey strategy / catalog selection does not
    depend on the redshift of the EM sources.  For example, a flux-limited
    survey of GW localization volumes would have a detection efficiency that is
    independent of $H_0$; a survey that focuses on objects with
    spectroscopically-measured redshifts consistent with the GW $d$ would have a
    selection function that is strongly-dependent on $H_0$.
\end{itemize}
With these assumptions, $P_{\detected,\EM}$ is varies only with $x$.  We propose
to use an expansion of $P_{\detected,\EM}$ in Legendre polynomials to model and
learn the EM selection function.  Since $P_{\detected,\EM}$ must always be
positive, we write
\begin{equation}
    \label{eq:Pdet-model}
    \log P_{\detected,\EM}\left( x, d, H_0 \right) = \sum_{l=1}^{N_l} A_l P_l(x)
\end{equation}
(The normalization of $P_{\detected,\EM}$ does not affect the likelihood; here
we have fixed the normalization by omitting the constant $P_0(x) = 1$ term from
the sum.)  The $A = \left\{ A_l \mid l = 1, \ldots, N_l \right\}$ are additional
parameters of our population model that we will infer.

Some anticipated inclination-dependent selection effects fit naturally into this
form (e.g.\ kilonovae are expected to have significant emission into all viewing
angles \todo{citations}); others do not (e.g.\ GRB detections require a small
viewing angle \todo{citations}).  We will show below that even for the latter
case, where $P_{\detected,\EM}$ varies strongly---even discontinuously---with
$x$, the smooth Legendre polynomial interpolation is sufficient to reduce
systematics below the percent-level in a realistic measurement.

\section{Examples}

\subsection{Fitting Weakly Variable EM Selection}
\label{sec:weakly-variable}

Let 
\begin{equation}
    \label{eq:weakly-variable-efficiency}
    \log P_{\detected,\EM}(x) = P_2(x) = \frac{1}{2}\left( 3 x^2 - 1 \right),
\end{equation}
and $N_l = 2$.  We generate a mock catalog of $N = 128$ GW+EM detections with
S/N $\rho > 10$, and $H_0 = 0.7$, and calculate the joint posterior on $H_0$, $A
= \left\{ A_1, A_2 \right\}$, and each event's $x$.  Figure
\ref{fig:inclination-P2} shows that we recover the intrinsic inclination
distribution of these events well.

\begin{figure}
    \script{inclination-P2.py}
    \includegraphics[width=\columnwidth]{figures/inclination-P2.pdf}
    \caption{The posterior on the EM detection efficiency inferred from 128 mock
    GW+EM detections with the EM detection efficiency described in \S\
    \ref{sec:weakly-variable}.  The blue line gives the posterior median; bands
    give the 68\% and 95\% credible intervals.  The black line is the true
    detection efficiency used to generate the mock data set (Eq.
    \eqref{eq:weakly-variable-efficiency}).}
    \label{fig:inclination-P2} 
\end{figure}

Ultimately, fitting the EM selection effects with this dataset results in
recovery of $H_0 = \variable{output/H0-P2.txt}$ (median and 68\% credible
interval, i.e.\ sub-percent uncertainty). Failing to account for the
inclination-dependent EM selection effect introduces significant bias in the
$H_0$ estimate, recovering $H_0 = \variable{output/H0-P2-nofit.txt}$
\citep{Chen2020}.  Figure \ref{fig:H0-P2} illustrates these effects.

\begin{figure}
    \script{H0-P2.py}
    \includegraphics[width=\columnwidth]{figures/H0-P2.pdf}
    \caption{Posterior on $H_0$ for the mock catalog discussed in \S\,
    \ref{sec:weakly-variable}.  The blue curve is the posterior when fitting
    $P_{\detected,\EM}$ with $N_l = 2$.  The fitted value is $H_0 =
    \variable{output/H0-P2.txt}$ (median and 68\% credible interval).  The
    orange curve is the posterior ignoring inclination-dependent EM selection
    effects, which results in $H_0 = \variable{output/H0-P2-nofit.txt}$, a $\sim
    2\sigma$ bias.  The black vertical line is the true $H_0$ used to generate
    the catalog.}
    \label{fig:H0-P2}
\end{figure}

\subsection{Fitting a Strongly Variable EM Selection}
\label{sec:strongly-variable}

Different EM counterparts will exhibit different dependence on inclination
angle.  In this subsection, we model an GW+EM catalog composed of two
components.  One component has EM detectable emission only very close to on/off
axis, with $P_{\detected,\EM} = 0$ unless $|x| \geq x_\mathrm{min}$ with
$x_\mathrm{min} = \cos 15^\circ$.  The other component has a (weak) preference
for on/off axis emission, with detectability varying quadratically from a
maximum at $|x| = 1$ to and 1/3 of its maximum value at $x = 0$.  The two
components are mixed equally in the data set, so that the overall EM detection
efficiency is 
\begin{equation}
    P_{\detected,\EM}(x) \propto \frac{1}{2} \left( \frac{H\left( |x| - x_\mathrm{min} \right)}{x_\mathrm{min}} + \frac{3 \left( x^2 + 1/2 \right)}{5} \right),
\end{equation}
with $H$ the Heaviside step function.  We choose $N_l = 4$.  The logarithm of
this function is not expressible as any finite sum of Legendre polynomials, so
our model is incapable of exactly reproducing the shape of this detection
efficiency for any choice of $N_l$.  Nonetheless, we will see that we can still
obtain sub-percent accurate, un-biased estimates of $H_0$, even with only four
additional parameters to describe the EM selection function.

Once again, we generate a catalog of $N = 128$ GW+EM mock detections with S/N
$\rho > 10$ using $H_0 = 0.7$.  We recover a $P_{\detected,\EM}$ shape that is
indicative of the EM detection efficiency, but the true efficiency lies well
outside the posterior 95\% credible interval over much of the range $-1 < x <
1$.  Figure \ref{fig:top-hat-Pdet} shows the posterior on $P_{\detected,\EM}$.

\begin{figure}
    \script{top-hat-Pdet.py}
    \includegraphics[width=\columnwidth]{figures/top-hat-Pdet.pdf}
    \caption{Posterior on $P_{\detected,\EM}$ from fitting the mock catalog
    described in \S\, \ref{sec:strongly-variable}.  The catalog is an equal
    mixture of an on/off axis component ($|x| > \cos 15^\circ$) and a
    quadratically-varying component; the true $P_{\detected,\EM}$ is shown by
    the black curve.  The blue line shows the posterior median, and the dark and
    light bands the posterior 68\% and 95\% credible intervals.  For most of the
    interval $-1 < x < 1$ the true detection efficiency is well outside the
    posterior 95\% credible interval due to the mismatch between the Legendre
    polynomial fitting function, Eq.\ \eqref{eq:Pdet-model} with $N_l = 4$, and
    the sharply-varying detection efficiency.}
    \label{fig:top-hat-Pdet}
\end{figure}

Even though the posterior on the detection efficiency does favors curves very
different to the true detection efficiency, the posterior on $H_0$ is not
biased.  The imperfect measurements of inclination from the GW data ``soften''
the bias in the recovery of the detection efficiency enough to recover the a
reasonable posterior on $H_0$.  This is shown in Figure \ref{fig:H0-top-hat},
which also shows that failing to account for the inclination-dependent selection
effects here leads to very significant bias in the recovered $H_0$ value.

\begin{figure}
    \script{H0-top-hat.py}
    \includegraphics[width=\columnwidth]{figures/H0-top-hat.pdf}
    \caption{Posterior on $H_0$ from fitting the mock catalog described in \S\,
    \ref{sec:strongly-variable}.  The vertical black line shows $H_0 = 0.7$, the
    value used to prepare the catalog.  The blue curve shows the posterior when
    fitting $P_{\detected,\EM}$ using Eq.\ \eqref{eq:Pdet-model} with $N_l=4$.
    In this case we find $H_0 = \variable{output/H0-top-hat.txt}$ (median and
    68\% credible interval).  The orange curve shows the posterior when fitting
    without accounting for inclination-dependent EM selection.  In this case we
    find $H_0 = \variable{output/H0-top-hat-flat.txt}$, with $\sim 8 \sigma$
    bias.}
    \label{fig:H0-top-hat}
\end{figure}

\begin{acknowledgments}
    \todo{We thank people for stuff.}  This work was begun at the KICP workshop
    ``The Quest for Precision Gravitational Wave Cosmology'' in September 2022;
    the authors thank the KICP and workshop organizers for an exceptionally
    stimulating environment.
\end{acknowledgments}

\software{\texttt{arviz} \citep{arviz_2019}, \texttt{matplotlib} \citep{Hunter:2007}, \texttt{pymc} \citep{salvatier2016probabilistic}, \texttt{seaborn} \citep{Waskom2021}, \texttt{showyourwork} \citep{Luger2021}}

\clearpage

\bibliography{bib}

\end{document}
